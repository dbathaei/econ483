% Options for packages loaded elsewhere
\PassOptionsToPackage{unicode}{hyperref}
\PassOptionsToPackage{hyphens}{url}
%
\documentclass[
]{article}
\usepackage{amsmath,amssymb}
\usepackage{iftex}
\ifPDFTeX
  \usepackage[T1]{fontenc}
  \usepackage[utf8]{inputenc}
  \usepackage{textcomp} % provide euro and other symbols
\else % if luatex or xetex
  \usepackage{unicode-math} % this also loads fontspec
  \defaultfontfeatures{Scale=MatchLowercase}
  \defaultfontfeatures[\rmfamily]{Ligatures=TeX,Scale=1}
\fi
\usepackage{lmodern}
\ifPDFTeX\else
  % xetex/luatex font selection
\fi
% Use upquote if available, for straight quotes in verbatim environments
\IfFileExists{upquote.sty}{\usepackage{upquote}}{}
\IfFileExists{microtype.sty}{% use microtype if available
  \usepackage[]{microtype}
  \UseMicrotypeSet[protrusion]{basicmath} % disable protrusion for tt fonts
}{}
\makeatletter
\@ifundefined{KOMAClassName}{% if non-KOMA class
  \IfFileExists{parskip.sty}{%
    \usepackage{parskip}
  }{% else
    \setlength{\parindent}{0pt}
    \setlength{\parskip}{6pt plus 2pt minus 1pt}}
}{% if KOMA class
  \KOMAoptions{parskip=half}}
\makeatother
\usepackage{xcolor}
\usepackage[margin=1in]{geometry}
\usepackage{graphicx}
\makeatletter
\def\maxwidth{\ifdim\Gin@nat@width>\linewidth\linewidth\else\Gin@nat@width\fi}
\def\maxheight{\ifdim\Gin@nat@height>\textheight\textheight\else\Gin@nat@height\fi}
\makeatother
% Scale images if necessary, so that they will not overflow the page
% margins by default, and it is still possible to overwrite the defaults
% using explicit options in \includegraphics[width, height, ...]{}
\setkeys{Gin}{width=\maxwidth,height=\maxheight,keepaspectratio}
% Set default figure placement to htbp
\makeatletter
\def\fps@figure{htbp}
\makeatother
\setlength{\emergencystretch}{3em} % prevent overfull lines
\providecommand{\tightlist}{%
  \setlength{\itemsep}{0pt}\setlength{\parskip}{0pt}}
\setcounter{secnumdepth}{-\maxdimen} % remove section numbering
\usepackage{booktabs}
\usepackage{longtable}
\usepackage{array}
\usepackage{multirow}
\usepackage{wrapfig}
\usepackage{float}
\usepackage{colortbl}
\usepackage{pdflscape}
\usepackage{tabu}
\usepackage{threeparttable}
\usepackage{threeparttablex}
\usepackage[normalem]{ulem}
\usepackage{makecell}
\usepackage{xcolor}
\ifLuaTeX
  \usepackage{selnolig}  % disable illegal ligatures
\fi
\usepackage[]{natbib}
\bibliographystyle{plainnat}
\IfFileExists{bookmark.sty}{\usepackage{bookmark}}{\usepackage{hyperref}}
\IfFileExists{xurl.sty}{\usepackage{xurl}}{} % add URL line breaks if available
\urlstyle{same}
\hypersetup{
  pdftitle={Will Your Event Sell Out? Answering a Prediction Problem Using Machine Learning Algorithms},
  pdfauthor={Daniel Bathaei; Amanda Muirhead},
  hidelinks,
  pdfcreator={LaTeX via pandoc}}

\title{Will Your Event Sell Out? Answering a Prediction Problem Using
Machine Learning Algorithms}
\usepackage{etoolbox}
\makeatletter
\providecommand{\subtitle}[1]{% add subtitle to \maketitle
  \apptocmd{\@title}{\par {\large #1 \par}}{}{}
}
\makeatother
\subtitle{ECON 483: Economic Applications of Machine Learning}
\author{Daniel Bathaei \and Amanda Muirhead}
\date{August 15, 2023}

\begin{document}
\maketitle

~~While by itself it is a piece of worthless paper, a ticket grants
individuals access to an event that is otherwise closed to the public.
If an event is well-liked, many people will buy tickets for it. If a
performer, sports team, or exhibition achieves fame and popularity, more
people might want to attend their events than are slots available. Once
every ticket for an event has been purchased, that event is sold out. To
performers, selling out is a sign of popularity. To producers, it is a
sign of a missed opportunity. They could have charged more for their
tickets, booked a bigger venue, or offered VIP packages. The ability to
accurately anticipate whether an event will hit full capacity can have
significant implications on a business' demand forecasting, market
analysis, and revenue optimization, not to mention the subsequent
effects on consumer spending and labour demand. Consequently, producers
are asking whether a set of variables such as price and location can
predict if their event will sell out. To this end, we use machine
learning models to predict whether a show will sell out.

~~The onset of machine learning in economics, with its capacity to
discern patterns from intricate data, has become an indispensable tool
in tackling complex prediction problems. Though it has several varying
definitions, the study of machine learning boils down to computer
programs' abilities to learn and adapt to new data without the
interference of humans (\citet{babenko2021}). To predict the likelihood
of an event selling out of tickets, we will employ several machine
learning techniques, namely Classification Trees, Logistic Regression,
Random Forests, Support Vector Machines, and Neural Networks.

\hypertarget{background-and-current-literature}{%
\section{Background and Current
Literature}\label{background-and-current-literature}}

~~Event tickets have been in the news a lot lately, namely for the
escalating rates at which they are being priced. Event-goers around the
world are increasingly lamenting over the rising costs of event tickets,
as reports come out of potential price gouging and fees as high as 78\%
of the original listed price (\citet{whyever2022}). While in 2018 it was
reported that ticket sales had doubled since the 1990's
(\citet{gigtick2018}), the percentage increase is now estimated to be
near triple their pre-Y2K levels (\citet{lastweektonight2022}), proving
that forecasting the demand for events is already one of many lucrative
strategies being employed by artists and ticket-distributing giants such
as Ticketmaster. Currently, event tickets sales are a \$78bn USD
industry, expected to show a growth of 9.7\% by 2024 (\citet{eventti}).

~~To contextualize our research, we studied existing literature
surrounding event sell-out and other related quantity demanded
predictions. Previous studies have illuminated various factors that
contribute to event success, encompassing variables such as ticket
pricing, marketing strategies, historical attendance data, and
socioeconomic indicators (\citet{krueger2005}).

\hypertarget{limitations}{%
\subsection{Limitations}\label{limitations}}

~~While writing this paper, we observed that companies work very hard to
conceal any information related to sales volume and quantity of tickets
demanded. StubHub has stopped providing businesses and researchers with
access to its application programming interface (API) and only provides
API services for producers for their own posted tickets\footnote{\url{https://developer.stubhub.com/api-reference/sales\#tag/Sales}}
Ticketmaster has recently removed any indicator of sales volume on their
API database. Companies like Pollstar\footnote{\url{https://www.pollstar.com/subscribe})}
charge considerable fees before providing any information on ticket
sales. This further solidifies the current importance and relevance of
ticket sale information.

\hypertarget{methodology}{%
\section{\texorpdfstring{\textbf{Methodology}}{Methodology}}\label{methodology}}

~~We use publicly available resources\footnote{Ethical obligations
  dictate that we only use publicly available data, even if we use
  scraping methods.} to prepare the data for our models. Pursuant to our
objective, we collected data from SeatGeek using the website's
application programming interface (API). Similar to other websites,
SeatGeek has reduced their available sales volume data. However,
SeatGeek still provides some insight on sales volume. Although the
company has announced their plan to move away from disclosing any sales
volume information through their API, this plan is yet to be
implemented.

\hypertarget{seatgeek}{%
\subsection{SeatGeek}\label{seatgeek}}

~~SeatGeek currently provides information regarding the volume of
tickets sold by an event, performer, and/or venue. This information is
shared through the `score' variable. The `score' variable is a
scaled\footnote{\(0 \le score \le 1\)} measure of an event, performer,
or venue's ticket sale volume relative to their type. This means that if
an event sells more tickets than any other event similar to its type,
that event will receive a score of 1. If an event makes no sales, that
event will have a score of 0. For each observation we can gather an
`event score', `venue score', and a performer score for each performer
in that event. This information is highly valuable since it is derived
from the quantity of units sold. Conveniently, SeatGeek analyzes the
social media following of each event's performers, and provides a
`popularity' value based on how popular an events performers are. This
value is also scaled between 0 and 1. Consequently, the variables we
gather from SeatGeek are shown on Table 1:

  \bibliography{references.bib}

\end{document}
